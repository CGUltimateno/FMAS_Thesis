\chapter{Introduction}
\label{chap:introduction}

Introduction

\section{The Company}

Description of the company.

\section{The Idea}

Introduction to the internship idea.

\section{Organization of the Text}

\begin{description}
    \item[{\hyperref[chap:literature-review]{The second chapter}}] describes ...
    
    \item[{\hyperref[chap:gap-analysis]{The third chapter}}] explores ...
    
    \item[{\hyperref[chap:system-analysis-and-design]{The fourth chapter}}] discusses ...
    
    \item[{\hyperref[chap:Methodology]{The fifth chapter}}] covers ...
    
    \item[{\hyperref[chap:conclusion]{The sixth chapter}}] examines ...
        
\end{description}

Regarding the drafting of the text, the following typographic conventions were adopted for this document:
\begin{itemize}
	\item Acronyms, abbreviations, and ambiguous or uncommon terms mentioned are defined in the glossary located at the end of this document;
    \item For the first occurrence of terms included in the glossary, the following nomenclature is used: (acronym);
	\item Terms in foreign languages or part of technical jargon are highlighted in \emph{italic}.
\end{itemize}
