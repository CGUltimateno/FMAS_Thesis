\chapter{System Analysis and Design}
\label{chap:system-analysis-and-design}

\intro{Brief introduction to the chapter}\\

\section{Use Cases}

For the study of the product's use cases, diagrams were created.
Use case diagrams (in English \emph{Use Case Diagram}) are a type of \gls{uml} diagram dedicated to describing the functions or services offered by a system, as perceived and used by the actors interacting with the system itself.
Since the project is aimed at creating a tool for automating a process, user interactions must obviously be reduced to the bare minimum. For this reason, the use case diagrams are simple and few in number.

\begin{figure}[!h] 
    \centering 

    \caption{Use Case - UC0: Main Scenario}
\end{figure}

\begin{usecase}{0}{Main Scenario}
\usecaseactors{Application Developer}
\usecasepre{The developer has entered the simulation plugin within the IDE}
\usecasedesc{The simulation window provides commands to configure, record, or run a test}
\usecasepost{The system is ready to allow a new interaction}
\label{uc:main-scenario}
\end{usecase}

\section{Requirements Traceability}

From a careful analysis of the requirements and use cases carried out on the project, a table was drawn up that traces the requirements in relation to the use cases.\\
Different types of requirements were identified, and an identification code was used to distinguish them.\\
The requirements code is structured as R(F/Q/V)(N/D/O) where:
\begin{enumerate}
    \item[R =] requirement
    \item[F =] functional
    \item[Q =] qualitative
    \item[V =] constraint
    \item[N =] mandatory (necessary)
    \item[D =] desirable
    \item[O =] optional
\end{enumerate}
Tables \ref{tab:functional-requirements}, \ref{tab:qualitative-requirements}, and \ref{tab:constraint-requirements} summarize the requirements and their traceability with the use cases outlined during the analysis phase.

\newpage

\begin{table}%
\caption{Table of Functional Requirements Traceability}
\label{tab:functional-requirements}
\begin{tabularx}{\textwidth}{lXl}
\hline\hline
\textbf{Requirement} & \textbf{Description} & \textbf{Use Case}\\
\hline
RFN-1     & The interface allows configuring the type of test probes & UC1 \\
\hline
\end{tabularx}
\end{table}%

\begin{table}%
\caption{Table of Qualitative Requirements Traceability}
\label{tab:qualitative-requirements}
\begin{tabularx}{\textwidth}{lXl}
\hline\hline
\textbf{Requirement} & \textbf{Description} & \textbf{Use Case}\\
\hline
RQD-1    & The performance of the hardware simulator must ensure the correct execution of tests and not generate false negatives & - \\
\hline
\end{tabularx}
\end{table}%

\begin{table}%
\caption{Table of Constraint Requirements Traceability}
\label{tab:constraint-requirements}
\begin{tabularx}{\textwidth}{lXl}
\hline\hline
\textbf{Requirement} & \textbf{Description} & \textbf{Use Case}\\
\hline
RVO-1    & The library for running automated tests must be reusable & - \\
\hline
\end{tabularx}
\end{table}%